\documentclass{ctexart}
\usepackage{graphicx}
\usepackage{amsmath}
\usepackage{amsthm}
\usepackage{amssymb}
\usepackage{fancyhdr}
\usepackage{ifthen}
\usepackage{syntonly}
\usepackage[colorlinks, CJKbookmarks=true, linkcolor=red]{hyperref}
\pagestyle{plain}
\usepackage[raggedright]{titlesec}
\newtheorem{性质}{性质}
\newtheorem{定理}{定理}
\newtheorem{推论}{推论}
\begin{document}
\title{作业}
\author{计算机科学与技术系52班 杨定澄 \and 学号:2015011274 \and E-mail:892431401@qq.com}
\date{}
\maketitle
我们设$\theta=\{x_{l1},x_{u1},x_{l2},x_{u2}\}$,就有
\begin{align*}
&P(x,y)=
\begin{cases}
\frac{1}{(x_{u1}-x_{l1})(x_{u2}-x_{l2})} & x_{l1} \le x \le x_{u1},x_{l2} \le y \le x_{u2} \\
0 & \textrm{otherwise} \\
\end{cases}
\\
&P(\theta|D^0)=\frac{1}{(x_{u1}-x_{l1})(x_{u2}-x_{l2})},\textrm{由数学归纳法可知}\\
&P(\theta|D^n)\propto	P(x_n|\theta)P(\theta|D^{n-1})\propto (\frac{1}{(x_{u1}-x_{l1})(x_{u2}-x_{l2})})^{n}\propto ((x_{u1}-x_{l1})(x_{u2}-x_{l2}))^{-n}\\
&P(x|D^n)=\int P(x|\theta)P(\theta|D^n) d\theta \propto \int \int \int \int ((x_{u1}-x_{l1})(x_{u2}-x_{l2}))^{-(n+1)}dx_{l1}dx_{u1}dx_{l2}dx_{u2}\\
\end{align*}

由于概率之和必定等于$1$,故比例系数完全可以反推出来,所以不做讨论。

接下来要做的是求解四重积分,这是非常麻烦的。观察到$(x_{l1},x_{u1})$与$(x_{l2},x_{u2})$可以分开考虑,我们的问题就变成求解下面的定积分:
\begin{align*}
&\int_{-6}^a\int_{b}^6(y-x)^{-(n+1)}dxdy \\
=& \int_{-6}^a \frac{-1}{n}(y-x)^{-n}\Big|_b^6 dx\\
=& \frac{1}{n}\int_{-6}^a [(b-x)^{-n}-(6-x)^{-n}]dx\\
=& \frac{1}{n(n-1)}[(b-x)^{-(n-1)}-(6-x)^{-(n-1)}]\Big|_{-6}^a\\
=& \frac{1}{n(n-1)}[(b-a)^{1-n}-(b+6)^{1-n}-(6-a)^{1-n}+12^{1-n}]
\end{align*}

至此,我们就能很容易的算出$P(x|D^n)$,每读入一组数据就画一个函数图像即可。

编译运行bayes.py,可以得到2.png,3.png,直到10.png,分别代表$n=2,3,\cdots,10$时的函数图像。
\end{document}
