\documentclass[12pt,a4paper]{article}
\usepackage{fontspec, xunicode, xltxtra}
\usepackage{xeCJK}
\usepackage{amsmath}
\usepackage{listings}
\usepackage{multirow}
\newfontfamily\courier{Courier}
\lstset{
  numbers=left,                   % where to put the line-numbers
  numberstyle=\tiny \courier,  % the style that is used for the line-numbers
  basicstyle=\courier,
  numbersep=2em,                  % how far the line-numbers are from the code
  breaklines=true,                % sets automatic line breaking
  breakatwhitespace=false,        % sets if automatic breaks should only happen at whitespace
}

\setCJKmainfont[BoldFont={Adobe Heiti Std R}, ItalicFont={Adobe Kaiti Std}]{Adobe Song Std L}
\setCJKmonofont{Courier}

\title{编译原理PA3作业报告}
\author{黄家晖 2014011330}
\date{}
\begin{document}
\maketitle

PA3利用Visitor模式对AST进行访问,利用部分静态语义信息生成TAC代码。
这种代码非常接近汇编代码,在此部分需要实现汇编语言的相关逻辑,增加新的语法特性。

\section{新增加的数据结构和函数}
\begin{enumerate}
\item 为了实现switch和case语句和\texttt{?:}操作符,需要为TranslatorPass2中遍历到相应节点时增加相应汇编语言生成。switch使用的逻辑是先经过若干个if条件判断应该前往哪一个分支,再在下面依次定义这些分支;而三元运算符使用的逻辑是先做条件判断,根据条件的真假跳转到不同的标签继续执行。
\item 为了实现\texttt{<<}操作符,为Class类增加CopyFuncLabel存储拷贝构造函数的Label信息,同时在VTable中存储对应的Class信息,将整个VTable的偏移量加4,第三个位置放置一个伪非静态的拷贝构造函数。
\item 为了实现repeat语句和对应的continue语句,新加入栈,存储当前所在循环的开始点,每次遇到continue语句就跳转到这个开始点继续执行。
\item 为了实现运行时检查的除0非法错误,需要在每次进行除法或是取模操作的时候,先插入条件判断代码,如果除数为0则终止程序,输出错误信息。
\end{enumerate}

\section{遇到的问题和解决方法}

本次PA我耗时比较长的是\texttt{<<}操作符的实现,一开始误认为最小公共父类可以利用静态语义信息得到(即使用上次PA中的分析结果),但改好代码之后,发现测例中所有的\texttt{<<}操作符都是运行时分析,而运行时分析的问题是不清楚新生成的对象大小是多少,应该如何拷贝信息。因此考虑定义拷贝构造函数(技术上来讲,定义一个getSize函数也可以,但是不如这种方式简便),这个拷贝构造函数是在这个阶段加入的新函数,与所有的函数都不同,既不是静态函数,它需要存在Vtable中,也不是非静态函数,它的第一个形参不是this而是需要拷贝的源对象地址。这个拷贝构造函数的功能是建立自己对应类的Vtable,并且拷贝源对象的部分函数到自己的地址空间来。这样,在遇到\texttt{<<}操作符的时候,只需要在运行时分析出最小的公共Vtable,即可查出所需拷贝构造函数进行隐式调用。

\end{document}
