\documentclass[12pt,a4paper]{article}
\usepackage{fontspec, xunicode, xltxtra}
\usepackage{xeCJK}
\usepackage{amsmath}
\usepackage{listings}
\usepackage{multirow}
\newfontfamily\courier{Courier}
\lstset{
  numbers=left,                   % where to put the line-numbers
  numberstyle=\tiny \courier,  % the style that is used for the line-numbers
  basicstyle=\courier,
  numbersep=2em,                  % how far the line-numbers are from the code
  breaklines=true,                % sets automatic line breaking
  breakatwhitespace=false,        % sets if automatic breaks should only happen at whitespace
}

\setCJKmainfont[BoldFont={Adobe Heiti Std R}, ItalicFont={Adobe Kaiti Std}]{Adobe Song Std L}
\setCJKmonofont{Courier}

\title{编译原理PA2作业报告}
\author{黄家晖 2014011330}
\date{}
\begin{document}
\maketitle

PA2主要实现静态语义检查,包括符号表的建立和类型的检查,为decaf编译器增加新的语法特性。

\section{新增加的数据结构和函数}
\begin{enumerate}
\item 在Error中新增加了错误类型,包括Continue语句不在循环体内的错误、选择三元运算符的算子不符合要求的错误、Case的标签不是整形常量错误等;
\item 在BuildSym的Visitor类中添加访问repeat语句和switch, case语句的符号表建立方法,对于switch来说,其作用域直接和外层相同(根据new\_test\_q3.decaf文件
的标准输出推断可知),而对于repeat语句,作用域为本地的新建作用域(参照C的语言规范);
\item 在TypeCheck的Visitor类中添加三元运算符的语义检查、continue的语义检查以及二元运算符\texttt{<<}的语义检查。

\end{enumerate}

\section{遇到的问题和解决方法}

对于建立符号表的过程,注意到实验文档中给出repeat的参考文法是\texttt{repeat Stmt* until (BoolExpr)},而只有Block才能关联本地作用域,
因此采取一个巧妙的做法,在建立repeat抽象语法树的时候就将\texttt{Stmt*}放在\texttt{Block}当中,实际上为AST嵌套了一层\texttt{Block}节点,
这样以来,repeat内的所有语句也能像if那样被看做一个单独的Scope进行处理了。

对于类型检查的过程,逻辑比较复杂的是三元运算符部分,代码中我尽量使用已经定义好的函数。首先判断条件语句是否为Bool型,如果不是则报错,但
不影响三元运算符语法节点的属性文法;接着看\texttt{?}后面的选择语句,如果二者是\texttt{compatible}的(这个函数考虑了更多的情况),且
都不是错误类型,那么返回左边的属性,如果有一个是错误类型,属性文法就整体为错误;如果二者不能\texttt{compatible},那么输出报错,属性文法为错误。

另外对二元运算符\texttt{<<}计算最小公共父类采用最简单的$O(n^2)$进行查找,详情请参见函数\texttt{TypeCheck.getMinCommonAncestor()}。


\end{document}
