% UTF-8 encoding
% Compile with latex+dvipdfmx, pdflatex, xelatex or lualatex

\documentclass[hyperref, UTF8]{ctexart}
\usepackage{graphicx}
\usepackage{amssymb}
\usepackage{amsmath}
\usepackage{subfigure}
\usepackage{geometry}
\usepackage{caption}

\newcommand{\true}{{\rm T}}
\newcommand{\false}{{\rm F}}
\newcommand{\zhyou}{\text{由}}
\newcommand{\zhfenli}{\text{分离}}
\newcommand{\zhguijie}{\text{归结}}
\newcommand{\zhqianti}{\text{前提}}
\newcommand{\zhzhihuan}{\text{置换}}
\newcommand{\zhdairu}{\text{代入}}
\newcommand{\zherqie}{\text{而且}}
\newcommand{\zhaxiom}{\text{公理}}
\newcommand{\zhtheorem}{\text{定理}}
\newcommand{\zhdefinition}{\text{定义}}

\title{离散数学——第七周作业}
\author{计83  刘轩奇  2018011025}
\date{2019.10.23}

\geometry{left=2.0cm, right=2.0cm, top=2.5cm, bottom=2.5cm}

\begin{document}

\maketitle

\paragraph{5.1}\label{5.1}
证明下列等值式和蕴含式

(6) $ ( \exists x)(P(x) \rightarrow Q(x)) = (al x)P(x) \rightarrow ( \exists x)Q(x) $

(8) $ ( \exists x)P(x) \land ( \forall x)Q(x) \Rightarrow ( \exists x)(P(x) \lor Q(x)) $

(10) $ ( \exists z) ( \exists y) ( \exists x)((P(x,z) \rightarrow Q(x,z)) \lor ((R(y,z) \rightarrow S(y,z)) $ 

\qquad $ = ( ( \forall z) ( \forall x)P(x,z) \rightarrow ( \exists z) ( \exists x)Q(x,y)) \lor ( ( \forall z) ( \forall y)R(x,y) \rightarrow ( \exists z) ( \exists y)S(y,z)) $

\paragraph{证}
(6)
\begin{align*}
( \exists x)(P(x) \rightarrow Q(x)) 
&= ( \exists x)(\lnot P(x) \lor Q(x))
\\ &= ( \exists x)(\lnot P(x)) \lor ( \exists x)(Q(x))
\\ &= \lnot ( \forall x)P(x) \lor ( \exists x)Q(x) 
\\ &= ( \forall x)P(x) \rightarrow ( \exists x)Q(x)
\end{align*}

(8)
\begin{align*}
( \exists x)P(x) \land ( \forall x)Q(x) 
&= ( \exists x)P(x) \land ( \forall y)Q(y)
\\ &= ( \exists x)(P(x) \land ( \forall y)Q(y))
\\ &\Rightarrow ( \exists x)(P(x) \land Q(x))
\end{align*}

(10)
\begin{align*}
& ( \exists z) ( \exists y) ( \exists x)((P(x,z) \rightarrow Q(x,z)) \lor ((R(y,z) \rightarrow S(y,z)) 
\\ & = ( \exists z) ( \exists y) ( \exists x)((P(x,z) \rightarrow Q(x,z)) \lor ( \exists z) ( \exists y) ( \exists x)(R(y,z) \rightarrow S(y,z))
\\ & = ( \exists z) ( \exists x)(P(x,z) \rightarrow Q(x,z)) \lor ( \exists z) ( \exists y)(R(y,z) \rightarrow S(y,z))
\\ & = ( \exists z) ( \exists x)(\lnot P(x,z) \lor Q(x,z)) \lor ( \exists z) ( \exists y)(\lnot R(y,z) \lor S(y,z)) 
\\ & = ( ( \exists z) ( \exists x) \lnot P(x,z) \lor ( \exists z) ( \exists x) Q(x,z)) \lor ( ( \exists z) ( \exists y) \lnot R(y,z) \lor ( \exists z) ( \exists y)S(y,z))
\\ & = (\lnot ( \forall z) ( \forall x)P(x,z) \lor ( \exists z) ( \exists x)Q(x,z)) \lor (\lnot ( \forall z) ( \forall y) R(y,z) \lor ( \exists z) ( \exists y)S(y,z))
\\ & = ( ( \forall z) ( \forall x)P(x,z) \rightarrow ( \exists z) ( \exists x)Q(x,y)) \lor ( ( \forall z) ( \forall y)R(x,y) \rightarrow ( \exists z) ( \exists y)S(y,z))
\end{align*}

\paragraph{5.2}\label{5.2}
判断下列各公式哪些是普遍有效的并给出证明,不是普遍有效的举出反例。

(3) $( ( \exists x)P(x) \rightarrow ( \forall x)Q(x)) \rightarrow )(\forall x)P(x) \rightarrow ( \forall x)Q(x)) $

(5) $ ( ( \exists x)P(x) \rightarrow ( \exists x)Q(x)) \rightarrow (\exists x)(P(x) \rightarrow Q(x)) $

(7) $ ( \exists x)P(x) \land ( \exists x)Q(x) \rightarrow ( \exists x)(P(x) \land Q(x)) $

\paragraph{解}
(3)
\begin{align*}
& ( ( \exists x)P(x) \rightarrow ( \forall x)Q(x)) \rightarrow ( \forall x)(P(x) \rightarrow Q(x))
\\ &= (\lnot ( \exists x)P(x) \lor ( \forall x)(Q(x)) \rightarrow ( \forall x)(P(x) \rightarrow Q(x))
\\ &= ( ( \forall x) \lnot P(x) \lor ( \forall x)Q(x)) \rightarrow ( \forall x)(P(x) \rightarrow Q(x))
\\ &= ( ( \forall x) \lnot P(x) \lor ( \forall x)Q(x)) \rightarrow ( \forall x)(\lnot P(x) \lor Q(x))
\\ &= \true
\end{align*}

则原公式普遍有效。

(5)
\begin{align*}
& ( ( \forall x)P(x) \rightarrow ( \exists x)Q(x)) \rightarrow ( \exists x)(P(x) \rightarrow Q(x))
\\ &= (\lnot ( \exists x)P(x) \lor ( \exists x)Q(x)) \rightarrow ( \exists x)(\lnot P(x) \lor Q(x))
\\ &= ( ( \forall x) \lnot P(x) \lor ( \exists x)Q(x)) \rightarrow ( \exists x)(\lnot P(x) \lor Q(x))
\\ &= ( ( \exists x) \lnot P(x) \rightarrow ( \exists x) \lnot P(x)) \lor ( \exists x)Q(x)
\\ &= \true \lor ( \exists x)Q(x)
\\ &= \true 
\end{align*}

则原公式普遍有效。

(7) 原公式并非普遍有效。例如在论域$\{1,2\}$上,令$P(1)=Q(2)=\true$, $P(2)=Q(1)=\false$, 则原公式真值为$\false$。

\end{document} 