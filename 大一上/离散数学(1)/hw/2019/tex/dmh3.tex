% UTF-8 encoding
% Compile with latex+dvipdfmx, pdflatex, xelatex or lualatex

\documentclass[hyperref, UTF8]{ctexart}
\usepackage{graphicx}
\usepackage{amssymb}
\usepackage{amsmath}
\usepackage{subfigure}
\usepackage{geometry}
\usepackage{caption}

\newcommand{\true}{{\rm T}}
\newcommand{\false}{{\rm F}}

\title{离散数学——第四周作业}
\author{计83  刘轩奇  2018011025}
\date{2019.09.28}

\geometry{left=2.0cm, right=2.0cm, top=2.5cm, bottom=2.5cm}
\linespread{0.7}
\begin{document}

\maketitle

\paragraph{2.5}\label{2.5}
给出下列公式的合取范式、析取范式、主合取范式和主析取范式。并给出所有使公式为真的解释。\\

(3) $ ( \lnot P \lor \lnot Q) \rightarrow (P \leftrightarrow \lnot Q) $ \\

(5) $ P \land (Q \lor ( \lnot P \land R)) $ \\

(7) $ P \rightarrow (Q \land ( \lnot P \leftrightarrow Q)) $ \\

\paragraph{解}

(3)
\begin{align*}
( \lnot P \lor \lnot Q) \rightarrow (P \leftrightarrow \lnot Q)
    &= \lnot ( \lnot P \lor \lnot Q) \lor (P \land \lnot Q) \lor ( \lnot P \land Q) \\
    &= (P \land Q) \lor (P \land \lnot Q) \lor ( \lnot P \land Q) & \text{(析取范式)} \\
    &= \lor_{1;2;3} & \text{(主析取范式)} \\
( \lnot P \lor \lnot Q) \rightarrow (P \leftrightarrow \lnot Q)
    &= \lnot ( \lnot P \lor \lnot Q) \lor ((P \lor Q) \land ( \lnot P \lor \lnot Q)) \\
    &= (P \land Q) \lor (( \lnot P \lor \lnot Q) \land (P \lor Q)) \\
    &= P \lor Q & \text{(合取范式)}\\
    &= \land _3 & \text{(主合取范式)}
\end{align*}

该式在$P=\true$或$Q=\true$时为真。\\

(5)
\begin{align*}
P \land (Q \lor ( \lnot P \land R))
    &= P \land (Q \lor \lnot P) \land (Q \lor R) & \text{(合取范式)} \\
    &= (P \lor (Q \land \lnot Q) \lor (R \land \lnot R)) \land (Q \lor \lnot P \lor (R \land \lnot R)) \\
    & \quad \land (Q \lor R \lor (P \land \lnot P)) \\
    &= (P \lor Q \lor R) \land (P \lor Q \lor \lnot R) \land (P \lor \lnot Q \lor R) \\
    & \quad \land (P \lor \lnot Q \lor \lnot R) \land ( \lnot P \lor Q \lor R) \land ( \lnot P \lor Q \lor \lnot R) \\
    & \quad \land (P \lor Q \lor R) \land ( \lnot P \lor Q \lor R) \\
    &= \land _{2;3;4;5;6;7} & \text{(主合取范式)}\\
    &= \lor _{6;7} & \text{(主析取范式)}\\
    &= (P \land Q) & \text{(析取范式)}
\end{align*}

该式在$P=\true$且$Q=\true$时为真。\\

(7)
\begin{align*}
P \rightarrow (Q \land ( \lnot P \leftrightarrow Q))
    &= \lnot P \lor (Q \land (( \lnot P \land Q) \lor (P \land \lnot Q))) \\
    &= \lnot P \lor (Q \land \lnot P \land Q) \lor (Q \land P \land \lnot Q) \\
    &= \lnot P \lor (Q \land \lnot P) \\
    &= \lnot P & \text{(合取范式、析取范式)} \\
    &= \lor _{0;1} & \text{(主析取范式)} \\
    &= \land _{0;1} & \text{(主合取范式)}
\end{align*}

该式在$P=\false$时为真。\\

\paragraph{2.6}\label{2.6}
分别以$A \to B$永真, $A\land \lnot B$永假以及解释法来证明下列各重言蕴含式$A\Rightarrow B$。\\

\paragraph{证}

(2) $ (P \rightarrow (Q \rightarrow R)) \Rightarrow (P \rightarrow Q) \rightarrow (P \rightarrow R) $ \\

\quad (a) $A \to B$永真
$$ (P \rightarrow (Q \rightarrow R)) \rightarrow ((P \rightarrow Q) \rightarrow (P \rightarrow R)) = (P \rightarrow (Q \rightarrow R)) \rightarrow (P \rightarrow (Q \rightarrow R)) = \true $$

\quad (b) $A\land \lnot B$永假 \\
$$ (P \rightarrow (Q \rightarrow R)) \land \lnot ((P \rightarrow Q) \rightarrow (P \rightarrow R)) = (P \rightarrow (Q \rightarrow R)) \land \lnot (P \rightarrow (Q \rightarrow R)) = \false $$

\quad (c) 解释法 \\
$$P \rightarrow (Q \rightarrow R) = \true $$

\qquad 若$P=\true$则$P \rightarrow Q = \true , (P \rightarrow R) = \true , (P \rightarrow Q) \rightarrow (P \rightarrow R) = \true $。

\qquad 若$P=\false$则$(P \rightarrow Q) \rightarrow (P \rightarrow R)= \true $。

\qquad 综上,蕴含式成立。\\

(4) $ (P \land Q) \rightarrow R \Rightarrow P \rightarrow (Q \rightarrow R) $ \\

\quad (a) $A \to B$永真
\begin{align*}
((P \land Q) \rightarrow R) \rightarrow (P \rightarrow (Q \rightarrow R))
    &= ( \lnot (P \land Q) \lor R) \rightarrow ( \lnot P \lor ( \lnot Q \lor R)) \\
    &= ( \lnot P \lor \lnot Q \lor R) \rightarrow ( \lnot P \lor \lnot Q \lor R) \\
    &= \true
\end{align*}

\quad (b) $A\land \lnot B$永假 \\
\begin{align*}
((P \land Q) \rightarrow R) \land \lnot (P \rightarrow (Q \rightarrow R))
    &= ( \lnot P \lor \lnot Q \lor R) \land \lnot ( \lnot P \lor \lnot Q \lor R) \\
    &= \false
\end{align*}

\quad (c) 解释法 \\
\begin{flalign*}
& (P \land Q) \rightarrow R= \true \\
& \therefore \lnot (P \land Q) \lor R= \true \\
& \therefore P \rightarrow ( \lnot Q \lor R) = \true \\
& \therefore P \rightarrow (Q \rightarrow R) = \true
\end{flalign*}

\paragraph{2.7}\label{2.7}
判断下列推理式是否正确?\\

(9) $ (P \land Q) \rightarrow R \Rightarrow (P \rightarrow R) \land (Q \rightarrow R) $ \\

(11) $ P \rightarrow Q \Rightarrow (P \rightarrow R) \rightarrow (Q \rightarrow R) $ \\

(13) $  \lnot (P \rightarrow Q) \land (Q \rightarrow P) \Rightarrow P \land \lnot Q $ \\

(15) $ (P \rightarrow Q) \land (R \rightarrow Q) \land (S \rightarrow Q) \Rightarrow (P \land R \land \lnot S \rightarrow Q) $ \\

\paragraph{解}

(9)
\begin{align*}
((P \land Q) \rightarrow R) \rightarrow ((P \rightarrow R) \land (Q \rightarrow R))
    &= \lnot ( \lnot (P \land Q) \lor R) \lor (( \lnot P \lor R) \land ( \lnot Q \lor R)) \\
    &= \lnot ( \lnot P \lor \lnot Q \lor R) \lor ( \lnot P \land Q) \lor R \\
    &= (P \land Q \land \lnot R) \lor R \lor ( \lnot P \land \lnot Q)
\end{align*}

当$R=\false, P=\true, Q=\false$时,上式为$\false$,则推理不正确。\\

(11)
\begin{align*}
(P \rightarrow Q) \rightarrow ((P \rightarrow R) \rightarrow (Q \rightarrow R))
    &= \lnot ( \lnot P \lor Q) \lor ( \lnot ( \lnot P \lor R) \lor ( \lnot Q \lor R)) \\
    &= (P \land \lnot Q) \lor (P \land \lnot R) \lor ( \lnot Q \lor R) \\
    &= (P \land \lnot Q) \lor \lnot Q \lor R \lor (P \land \lnot R) \\
    &= \lnot Q \lor R \lor (P \land \lnot R)
\end{align*}

当$P=\false, Q=\true, R=\false$时,上式为$\false$,则推理不正确。\\

(13)
设 $\lnot (P \rightarrow Q) \land (Q \rightarrow P)= \true $

\qquad 则 $P \rightarrow Q= \false , Q \rightarrow P= \true $

\qquad 由 $P \rightarrow Q= \false$ 知,$P= \true , Q= \false $

\qquad 则 $P\land \lnot Q=\true$,推理正确。\\

(15) 只要验证 $(P \rightarrow Q) \land (R \rightarrow Q) \land (S \rightarrow Q) \land (P \land R \land \lnot S) \Rightarrow Q$ 即可。

\qquad $P \rightarrow Q = \true, R \rightarrow Q = \true, S \rightarrow Q = \true, P \land R \land \lnot S = \true$ 即 $P=R=\true, S=\false$

\qquad 由 $P \rightarrow Q = \true, P = \true$ 知 $Q = \true$,从而推理正确。

\paragraph{2.8}\label{2.8}
使用推理规则证明 \\

(4) $ P \lor Q \rightarrow R \land S, S \lor E \rightarrow U \Rightarrow P \rightarrow U $ \\

(6) $  \lnot Q \lor S, (E \rightarrow \lnot U) \rightarrow \lnot S \Rightarrow Q \rightarrow E $ \\

\paragraph{证}

(4) 
\begin{align*}
(a) & (P \lor Q) \rightarrow (R \land S) & \text{前提引入} \\ 
(b) & (S \lor E) \rightarrow U & \text{前提引入} \\ 
(c) & P & \text{附加前提引入} \\ 
(d) & P \lor Q & (c)\text{置换} \\ 
(e) & R \land S & (a)(d)\text{分离} \\
(f) & S & (e)\text{置换} \\ 
(g) & S \lor E & (f)\text{置换} \\ 
(h) & U & (b)(g)\text{分离} \\
(i) & P \rightarrow U & \text{条件证明规则}
\end{align*}

(6)
\begin{align*}
(a) & \lnot (Q \lor S) & \text{前提引入} \\
(b) & (E \rightarrow \lnot U) \rightarrow \lnot S & \text{前提引入} \\
(c) & Q & \text{附加前提引入} \\
(d) & Q \rightarrow S & (a)\text{置换} \\
(e) & S & (c)(d)\text{分离} \\
(f) & S \rightarrow \lnot (E \rightarrow \lnot U) & (b)\text{置换} \\
(g) & \lnot (E \rightarrow \lnot U) & (e)(f)\text{分离} \\
(h) & E \land U & (g)\text{置换} \\
(i) & E & (h) \\
(j) & Q \rightarrow E & \text{条件证明规则}
\end{align*}

\end{document} 