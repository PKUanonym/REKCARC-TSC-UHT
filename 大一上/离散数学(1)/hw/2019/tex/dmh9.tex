% UTF-8 encoding
% Compile with latex+dvipdfmx, pdflatex, xelatex or lualatex

\documentclass[UTF8]{ctexart}
\usepackage{graphicx}
\usepackage{amssymb}
\usepackage{amsmath}
\usepackage{subfigure}
\usepackage{geometry}
\usepackage{caption}

\newcommand{\true}{{\rm T}}
\newcommand{\false}{{\rm F}}
\newcommand{\snatural}{\mathbb{N}}
\newcommand{\sinteger}{\mathbb{Z}}
\newcommand{\srational}{\mathbb{Q}}
\newcommand{\sreal}{\mathbb{R}}
\newcommand{\card}{\rm{card}}
\newcommand{\modt}{\text{mod }}

\title{离散数学——第十周作业}
\author{计83  刘轩奇  2018011025}
\date{2019.11.15}

\geometry{left=2.0cm, right=2.0cm, top=2.5cm, bottom=2.5cm}

\begin{document}

\maketitle

\paragraph{9.13} \label{9.13}
给定$\snatural$的下列子集,$A,B,C,D$为 
$A = \{ 1,2,7,8 \},$
$B = \{ x | x^2 < 50 \} ,$
$C = \{ x|0 \le x \le 20 \land x \text {可被} 3\text {整除} ,$
$D = \{ x|x=2^K \land K \in \snatural \land 0 \le K \le 5 \}.$
列出下列集合的所有元素。

(3) $B-(A \cup C)$

\paragraph{解}
(3) $B - (A \cup C) = \{ 0,1,2,3,4,5,6,7 \} - \{ 0,1,2,3,6,7,8,9,12,15,18 \} = \{ 4,5 \} $

\paragraph{9.14} \label{9.14}
写出下列集合:

(1) $\cup \{ \{ 3,4 \} , \{ \{ 3 \} , \{ 4 \} \} , \{ 3, \{ 4 \} \} , \{ \{ 3 \} ,4 \} \}$

(2) $\cap \{ \{ 1,2,3 \} , \{ 2,3,4 \} , \{ 3,4,5 \} \} $

\paragraph{解}
(1) $\cup \{ \{ 3,4 \} , \{ \{ 3 \} , \{ 4 \} \} , \{ 3, \{ 4 \} \} , \{ \{ 3 \} ,4 \} \} =  \{ 3,4, \{ 3 \} , \{ 4 \} \} $

(2) $\cap \{ \{ 1,2,3 \} , \{ 2,3,4 \} , \{ 3,4,5 \} \}  = \{ 3 \}$

\paragraph{9.15} \label{9.15}
写出下列集合,其中:$PP(A)=P(P(A)), PPP(A)=P(P(P(A)))$

(1) $ \cup \{ PPP(\varnothing), PP(\varnothing), P(\varnothing), \varnothing \} $

(2) $ \cap \{ PPP(\varnothing), PP(\varnothing), P(\varnothing) \} $

\paragraph{解}
$$P(\varnothing) = \{ \varnothing \} , P(P(\varnothing)) = \{ \varnothing, \{ \varnothing \} \} , P(P(P(\varnothing))) = \{ \varnothing, \{ \varnothing \} , \{ \{ \varnothing \} \} , \{ \varnothing, \{ \varnothing \} \} \} $$

(1) $ \cup \{ PPP(\varnothing), PP(\varnothing), P(\varnothing), \varnothing \} = \{ \varnothing, \{ \varnothing \} , \{ \{ \varnothing \} \} , \{ \varnothing, \{ \varnothing \} \} \} $

(2) $ \cap \{ PPP(\varnothing), PP(\varnothing), P(\varnothing) \} = \{ \varnothing \} $

\paragraph{9.16} \label{9.16}
设$ A = \{ \{ \varnothing \}, \{ \{ \varnothing \} \} \} $,写出集合:

(1) $ P(A) $ 和 $ \cup P(A) $

(2) $ \cup A $ 和 $ P( \cup A) $

\paragraph{解}
(1) $ P(A) = \{ \varnothing, \{ \{ \varnothing \} \} , \{ \{ \{ \varnothing \} \} \} , \{ \{ \varnothing \} , \{ \{ \varnothing \} \} \} \} , \cup P(A) = \{ \{ \varnothing \} , \{ \{ \varnothing \} \} \} $

(2) $ \cup (A) = \{ \varnothing, \{ \varnothing \} \} , P( \cup (A)) = \{ \varnothing, \{ \varnothing \} , \{ \{ \varnothing \} \} , \{ \varnothing, \{ \varnothing \} \} \} $

\paragraph{9.17} \label{9.17}
设$A,B,C$是任意的集合,证明:

(2) $(A-B)-C=(A-C)-(B-C)$

(4) $A\subseteq C \land B\subseteq C \Leftrightarrow A \cup B\subseteq C$

\paragraph{解}
(2)
\begin{align*}
    & (A-C) - (B-C) \\
    = & (A \cup -C) \cup -(B \cup -C) \\
    = & (A \cup -C) \cup (-B \cap C) \\
    = & (A \cup -C \cup -B) \cap (A \cup -C \cup C) \\
    = & (A \cup -B \cup -C) \cap (A \cup E) \\
    = & ((A-B) - C) \cap E \\
    = & (A-B)-C
\end{align*}
(4) 充分性:
$$A\subseteq C \land B\subseteq C \Rightarrow (A \cup B)\subseteq (C \cup C) \Rightarrow A \cup B\subseteq C$$
必要性:若$A \cup B \subseteq C$

任意$x$,若$x \in A$,则$x \in A \cup B$,从而$x \in C$,则$A \subseteq C$。

任意$x$,若$x \in B$,则$x \in A \cup B$,从而$x \in C$,则$B \subseteq C$。
$$\therefore A \cup B\subseteq C \Rightarrow A\subseteq C \land B\subseteq C$$

\paragraph{9.18} \label{9.18}
满足下列条件的集合$A,B$有什么关系?

(1) $A-B=B$

(3) $A \cap B = A \cup B$

\paragraph{答}
(1) $A=B=\varnothing$

(3) $A=B$

\paragraph{9.19} \label{9.19}
给出下列命题成立的充要条件:

(2) $ (A-B) \cup (A-C) = \varnothing $

(4) $ (A-B) \oplus (A-C) = \varnothing $

\paragraph{解}
(2)
\begin{align*}
     & (A-B) \cup (A-C) = \varnothing \\
    \Longleftrightarrow & (A \cup -B) \cap (A \cup -C) = \varnothing \\
    \Longleftrightarrow & A \cap -(B \cap C) = \varnothing \\
    \Longleftrightarrow & A\subseteq B \cap C
\end{align*}
(4)
\begin{align*}
    & (A-B) \oplus (A-C) = \varnothing \\
    \Longleftrightarrow & ((A-B)-(A-C) \cup ((A-C)-(A-B)) = \varnothing \\
    \Longleftrightarrow & ((A \cap -B) \cap -(A \cap -C)) \cup ((A \cap -C) \cap -(A \cap B)) = \varnothing \\
    \Longleftrightarrow & ((A \cap -B) \cap (-A \cup C)) \cup ((A \cap -C) \cap (-A \cup B)) = \varnothing \\
    \Longleftrightarrow & (A \cap -A \cap -B) \cup (A \cap -B \cap C) \cup (A \cap -A \cap -C) \cup (A \cap B \cap -C) = \varnothing \\
    \Longleftrightarrow & \varnothing \cup (A \cap -B \cap -C) \cup \varnothing \cup (A \cap B \cap -C) = \varnothing \\
    \Longleftrightarrow & A \cap -B \cap C = \varnothing \land A \cap B \cap -C = \varnothing \\
    \Longleftrightarrow & A-B\subseteq -C \land A-C\subseteq -B \\
    \Longleftrightarrow & A-B\subseteq A-C \land A-C\subseteq A-B & (\ast)\text{待后补证} \\
    \Longleftrightarrow & A-B = A-C
\end{align*}
补证(*)$A-B\subseteq -C \Longleftrightarrow A-B\subseteq A-C$
\begin{align*}
    A-B\subseteq -C 
    \Longleftrightarrow & ( \forall x) (x \in A-B \rightarrow x \in -C) \\
    \Longleftrightarrow & ( \forall x)((x \in A \land x \in -B) \rightarrow x \in C) \\
    \Longleftrightarrow & ( \forall x)((x \in A \land x \in -B) \rightarrow (x \in A \land x \in -C) \\
    \Longleftrightarrow & ( \forall x)(x \in A-B \rightarrow x \in A-C) \\
    \Longleftrightarrow & A-B\subseteq A-C
\end{align*}

\paragraph{9.26} \label{9.26}
(1) 若 $A \times B = \varnothing$ ,则$A$和$B$应满足什么条件。

(2) 对集合 $A$ ,是否可能$A = A \times A$。

\paragraph{答}
(1) $A \times B = \varnothing \Longleftrightarrow A=\varnothing \lor B = \varnothing$

(2) $A = A \times A \Longleftrightarrow A = \varnothing$

\paragraph{9.28} \label{9.28}
求$1$至$250$之间被$2,3,5$中任何一个整除的整数的个数。

\paragraph{解}
设
\begin{align*}
    E & = \{x|x \in \snatural \land 1\le x \le 250\} \\
    A & = \{x|x \equiv 0 (\modt 2) \land x \in E\} \\
    B & = \{x|x \equiv 0 (\modt 3) \land x \in E\} \\
    C & = \{x|x \equiv 0 (\modt 5) \land x \in E\} 
\end{align*}
$$|A|=\lfloor 250/2 \rfloor = 125, |B|=\lfloor 250/3 \rfloor = 83, |C|=\lfloor 250/5 \rfloor = 50 $$
$$|A \cap B|=\lfloor 250/6 \rfloor = 83, |A\cap C|=\lfloor 250/6 \rfloor = 25, |B\cap C|=\lfloor 250/15 \rfloor = 16 $$
$$|A \cap B \cap C|=\lfloor 250/30 \rfloor = 8 $$
\begin{align*}
    \therefore |A \cup B \cup C| & = |A| + |B| + |C| - |A \cup B| - |A \cup C| - |B \cup C| + |A \cup B \cup C| \\ & =  125+83+50-41-25-16+8 = 184
\end{align*}

\paragraph{9.30} \label{9.30}
证明不存在集合$A_1, A_2, A_3, A_4$使
$$A_4 \in A_3 \land A_3 \in A_2 \land A_2 \in A_1 \land A_1 \in A_4$$

\paragraph{证}
假设这样的$A_1, A_2, A_3, A_4$存在,则令$B=\{A_1, A_2, A_3, A_4\}$

由正则公理,$B$含有极小项,由对称性不妨设该极小项为$A_1$。则$A_1 \cap B = \varnothing$。

而$A_2 \in A_1, A_2 \in B$,则$A_2 \in A \cap B$,这与$A_1 \cap B = \varnothing$矛盾。

则假设错误,不存在这样的四个集合。

\paragraph{9.31} \label{9.31}
证明不存在由所有单元素集合组成的集合。

\paragraph{证}
假设存在所有单元素集合组成的集合$A$。

则令$B=\{A\}$,$B$是单元素集合,则$B \in A$,而又有$A \in B$,这与定理9.7.7: $\lnot (A_1 \in A_2 \land A_2 \in A_1)$矛盾。

故假设错误,不存在由所有单元素集合组成的集合。

\paragraph{9.32} \label{9.32}
证明存在所有素数组成的集合。

\paragraph{证}
由无穷公理知自然数集$\snatural$存在,设谓词$P(x)$表示$x$为素数,则由子集公理
$$(\exists A)(\forall x)(x \in A \longleftrightarrow x \in \snatural \land P(x))$$
即存在$A=\{x|x\text{为素数}\land x \in \snatural\}$为全体素数组成的集合。

\paragraph{9.33} \label{9.33}
证明若$A$是传递集合,则$A_+$是传递集合。

\paragraph{证}
\begin{align*}
    A\text{为传递集} 
    & \Longleftrightarrow ( \forall x) ( \forall y)((x \in y \land y \in A) \rightarrow x \in A) \\
    A^+ = A \cup \{ A \} & \Longrightarrow ( \forall x)(x \in A \rightarrow x \in A^+) \\
    & \Longleftrightarrow ( ( \forall x) ( \forall y)((x \in y \land y = A) \rightarrow x \in A^+) 
\end{align*}
\begin{align*}
    & A\text{为传递集}  \land A^+ = A \cup \{ A \} \\
    \Longrightarrow & ( ( \forall x) ( \forall y)((x \in y \land y \in A) \rightarrow x \in A^+) \land ( \forall x) ( \forall y)((x \in y \land y \in A) \rightarrow x \in A^+) \\
    \Longrightarrow & ( \forall x) ( \forall y)(((x \in y \land y \in A) \rightarrow x \in A^+) \land (((x \in y \land y \in A) \rightarrow x \in A^+)) \\
    \Longrightarrow & ( \forall x) ( \forall y)((x \in y) \land (y \in A \lor y = A)) \rightarrow x \in A^+) \\
    \Longrightarrow & ( \forall x) ( \forall y)((x \in y \land y \in A^+) \rightarrow x \in A^+) \\
    \Longrightarrow & A^+\text{为传递集}  
\end{align*}

\end{document} 