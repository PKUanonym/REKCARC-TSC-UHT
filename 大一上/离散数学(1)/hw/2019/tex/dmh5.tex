% UTF-8 encoding
% Compile with latex+dvipdfmx, pdflatex, xelatex or lualatex

\documentclass[hyperref, UTF8]{ctexart}
\usepackage{graphicx}
\usepackage{amssymb}
\usepackage{amsmath}
\usepackage{subfigure}
\usepackage{geometry}
\usepackage{caption}

\newcommand{\true}{{\rm T}}
\newcommand{\false}{{\rm F}}
\newcommand{\zhyou}{\text{由}}
\newcommand{\zhfenli}{\text{分离}}
\newcommand{\zhguijie}{\text{归结}}
\newcommand{\zhqianti}{\text{前提}}
\newcommand{\zhzhihuan}{\text{置换}}
\newcommand{\zhdairu}{\text{代入}}
\newcommand{\zherqie}{\text{而且}}
\newcommand{\zhaxiom}{\text{公理}}
\newcommand{\zhtheorem}{\text{定理}}
\newcommand{\zhdefinition}{\text{定义}}

\title{离散数学——第六周作业}
\author{计83  刘轩奇  2018011025}
\date{2019.10.18}

\geometry{left=2.0cm, right=2.0cm, top=2.5cm, bottom=2.5cm}

\begin{document}

\maketitle

\paragraph{3.3}\label{3.3}
依自然演绎系统证明

(3) $ A \rightarrow B, A \rightarrow \lnot B \vdash \lnot A $

\paragraph{证} (3) 
\begin{align*}
(a) & A \rightarrow B, A \rightarrow \lnot B, A \vdash A & (\text{肯定前提}) \\
(b) & A \rightarrow B, A \rightarrow \lnot B, A \vdash A \rightarrow B & (\text{肯定前提}) \\
(c) & A \rightarrow B, A \rightarrow \lnot B, A \vdash A \rightarrow \lnot B & (\text{肯定前提}) \\
(d) & A \rightarrow B, A \vdash B & (\text{分离规则}) \\
(e) & A \rightarrow \lnot B, A \vdash \lnot B & (\text{分离规则}) \\
(f) & A \rightarrow B, A \rightarrow \lnot B, A \vdash B & ((a)(b)(d)\text{传递律}) \\
(g) & A \rightarrow B, A \rightarrow \lnot B, A \vdash \lnot B & ((a)(c)(e)\text{传递律}) \\
(h) & A \rightarrow B, A \rightarrow \lnot B \vdash \lnot A & ((f)(g)\text{反证律})
\end{align*}

\paragraph{4.5}\label{4.5}
将下列语句符号化

(2) 凡有理数都可写为分数。

(5) 过平面上两个点,有且仅有一条直线通过。

(6) 凡实数都能比较大小。

(7) 在北京工作的人未必都是北京人。

(8) 只有一个北京。

(10) 如果明天天气好,有些学生将去香山。

\paragraph{答}

(2) $P(x): $ $x$是有理数;$Q(x): $ $x$可写成分数。

语句可化为$ ( \forall x)(P(x) \rightarrow Q(x))$。

(5) $P(x): $ $x$是平面上的点;$Q(x): $ $x$是直线;$R(x,y): $ $x$通过$y$。

语句可化为$ ( \forall x) ( \forall y)((P(x) \land P(y) \land \lnot E(x,y))$

(6) $P(x): $ $x$为实数;$Q(x,y): $ $x,y$能比较大小。

语句可化为$ ( \forall x) ( \forall y)((P(x) \land P(y)) \rightarrow Q(x,y))$

(7) $P(x): $ $x$是人;$Q(x): $ $x$在北京工作;$R(x): $ $x$是北京人。

语句可化为$ \lnot ( \forall x)(P(x) \land Q(x) \rightarrow R(x))$

(8) $P(x):$ $x$是北京;$E(x,y): $ $x,y$相同。

语句可化为$ ( \exists x)(P(x) \land ( \forall y)(P(x) \rightarrow E(x,y)))$

(10) $P: $ 明天天气好;$Q(x): $ $x$是学生;$R(x): $ $x$去香山。

语句可化为$ P \rightarrow ( \exists x)(Q(x) \land R(x))$

\paragraph{4.6}\label{4.6}
设$P(x)$表示x是有理数,$Q(x)$表示$x$是实数,$R(x)$表示$x$是无理数,$L(x)$表示$x$是正整数,$S(x)$表示$x$是偶数,$W(x)$表示$x$是奇数,试将下列公式翻译成自然语句。

(3) $  \lnot ( \forall x )(Q(x) \rightarrow P(x)) $ 

(6) $( \forall x ) (L(x) \rightarrow S(x) \bar \lor W(x)))$ 

(9) $ ( \forall x ) (L(x) \rightarrow P(x)) \land \lnot ( \forall x) (P(x) \rightarrow I(x)) $

\paragraph{答}
(3) 不是所有实数都是有理数。

(6) 任意正整数不是偶数就是奇数。

(9) 任意正整数都是有理数,但并非所有有理数都是正整数。

\paragraph{4.7}\label{4.7}
设个体域为$  \{ a, b, c \}  $,试将下列公式写成命题逻辑公式 

(3) $( \forall x )P(x) \land ( \exists x ) Q(x) $

(4) $( \forall x ) (P(x) \rightarrow Q(x))$

(5) $( \forall x ) \lnot P(x) \lor ( \forall x ) P(x)$

\paragraph{答}
(3) $ P(a) \land P(b) \land P(c) \land (Q(a) \lor Q(b) \lor Q(c)) $

(4) $ (P(a) \rightarrow Q(a)) \land (P(b) \rightarrow Q(b)) \land (P(c) \rightarrow Q(c)) $

(5) $ ( \lnot P(a) \lor P(a)) \land ( \lnot P(b) \lor P(b)) \land ( \lnot P(c) \lor P(c)) $

\paragraph{4.8}\label{4.8}
判断下列公式是普遍有效的,不可满足的还是可满足的?

(1) $  ( \forall x)P(x) \rightarrow P(y) $ 

(2) $  ( \exists x)(P(x) \land Q(x)) \rightarrow ( ( \exists x)P(x) \land ( \exists x)Q(x)) $ 

(3) $  ( \forall x)P(x) $ 

(4) $ ( \exists x)(P(x) \land \lnot P(x)) $ 

(5) $  ( \forall x)(P(x) \rightarrow Q(x)) $ 

(6) $  ( \forall x)(P(x) \lor \lnot P(x)) $ 

(7) $ ( ( \exists x)P(x) \land ( \exists x)Q(x)) \rightarrow ( \exists x)(P(x) \land Q(x)) $

\paragraph{答} (1) 普遍有效; (2) 普遍有效; (3) 可满足; (4) 不可满足; (5) 可满足; (6) 普遍有效; (7) 可满足。

\paragraph{4.9}\label{4.9}
给出一个公式,使其在$  \{ 1, 2 \}  $上是可满足的,而在$  \{ 1 \} $上是不可满足的。

\paragraph{答} $P(x): x=2$, 公式: $(\exists x)P(x)$

\paragraph{4.10}\label{4.10}
设个体域为$ \{ a,b \} $,并对$P(x,y)$设定为$P(a,a)=\true$, $P(a,b)=\false$, $P(b,a)=\false$, $P(b,b)=\true$,计算下列公式的真值。

(3) $  ( \forall x) ( \forall y)P(x,y) $ 

(5) $  ( \exists y) \lnot P(a,y) $ 

(7) $  ( \forall x) ( \forall y)(P(x,y) \rightarrow P(y,x)) $

\paragraph{解}
(3) $x=a, y=b, P(x,y)=\false$

$ \therefore (\forall x)(\forall y)P(x,y)=\false$

(5) $ y=b, \lnot P(a,y)=\true $

$\therefore (\exists y)\lnot P(a,y)=\true $

(7) 
\begin{gather*}
x=a, y=a, P(x,y)=P(y,x)= \true , P(x,y) \rightarrow P(y,x)= \true \\
x=a, y=b, P(x,y)=P(y,x)= \false , P(x,y) \rightarrow P(y,x)= \true \\
x=b, y=a, P(x,y)=P(y,x)= \false , P(x,y) \rightarrow P(y,x)= \true \\
x=b, y=b, P(x,y)=P(y,x)= \true , P(x,y) \rightarrow P(y,x)= \true \\
\therefore ( \forall x) ( \forall y)(P(x,y) \rightarrow P(y,x))= \true
\end{gather*}

\end{document} 