% UTF-8 encoding
% Compile with latex+dvipdfmx, pdflatex, xelatex or lualatex

\documentclass[hyperref, UTF8]{ctexart}
\usepackage{graphicx}
\usepackage{amssymb}
\usepackage{amsmath}
\usepackage{subfigure}
\usepackage{geometry}
\usepackage{caption}

\title{离散数学——第一周作业}
\author{计83  刘轩奇  2018011025}
\date{2019.09.20}

\geometry{left=2.0cm, right=2.0cm, top=2.5cm, bottom=2.5cm}
\linespread{0.7}
\begin{document}

\maketitle

\paragraph{1.1}\label{1.1}
判断下列语句是否是命题,并对命题确定其真值。\\

\quad (1) 火星上有声明存在。 \\

\quad (3) 香山比华山高。\\

\quad (5) 这盆茉莉花真香!\\

\quad (7) 这句话是错的。\\

\paragraph{答}
(1) 是命题。但真值尚未确定。(3) 是命题,真值为假。(5) 不是命题。感叹句不是命题。(7) 不是命题。悖论不是命题。

\paragraph{1.2}\label{1.2}
$P$表示今天很冷,$Q$表示正在下雪。\\

\quad (1) 将下列命题符号化:\\

\qquad 如果正在下雪,那么今天很冷。\\

\qquad 今天很冷当且仅当正在下雪。\\

\qquad 正在下雪的必要条件是今天很冷。\\

\quad (2) 用自然语句叙述下列公式:\\

\qquad $$\lnot(P\lor Q), \lnot P \land \lnot Q, P \to Q,
\lnot P \lor Q, \lnot \lnot P, \lnot P \leftrightarrow Q. $$

\paragraph{答}
\subparagraph{(1)}
$Q\to P, P \leftrightarrow Q, Q\to P.$
\subparagraph{(2)}
今天很冷与正在下雪两件事不同时发生;
今天不冷或没在下雪;
如果今天很冷,则正在下雪;
今天不冷或正在下雪;
今天并非不冷;
今天不冷当且仅当正在下雪。

\paragraph{1.3}\label{1.3}
对下列公式直观叙述在什么样的解释下为真,并列写出真值表来验证。\\

\quad (1) $ \lnot (P \lor Q), \lnot P \land Q, \lnot (P \land Q). $ \\

\quad (3) $ (P \to Q) \land (P \leftrightarrow Q). $ \\

\quad (5) $ P \to (Q \to R), P \land Q \to R.$ \\

\paragraph{解} 
\subparagraph{(1)}
$\lnot (P \lor Q)$和$ \lnot P \land Q$仅在$P$与$Q$均为$0$时真值为$1$。
$\lnot (P \land Q)$只需$P, Q$中有一个为$0$,真值即为$1$。真值表如表 1.3(1) 所示。
\begin{table}[!htb]
\centering
\begin{tabular}{ccccc}
$P$ & $Q$ & $\lnot(P \lor Q)$ & $\lnot P \land \lnot Q$ & $\lnot (P \land Q)$ \\
0 & 0 & 1                 & 1                       & 1                                               \\
0 & 1 & 0                 & 0                       & 1                                               \\
1 & 0 & 0                 & 0                       & 1                                               \\
1 & 1 & 0                 & 0                       & 0
\end{tabular}
\caption*{表 1.3 (1)}
\end{table}

\subparagraph{(3)}仅当$P$假$Q$真时,公式真值为$1$。真值表如表 1.3(3) 所示。
\begin{table}[!htb]
\centering
\begin{tabular}{ccc}
$P$ & $Q$ & $(P\to Q)\land \lnot (P \leftrightarrow Q)$ \\
0 & 0 & 0                                           \\
0 & 1 & 1                                           \\
1 & 0 & 0                                           \\
1 & 1 & 0
\end{tabular}
\caption*{表 1.3 (3)}
\end{table}

\subparagraph{(5)}上述公式仅在$P=Q=1, R=0$时真值为$0$,否则为$1$。真值表如图 1.3(5) 所示。
\begin{table}[!htb]
\centering
\begin{tabular}{ccccc}
$P$ & $Q$ & $R$ & $P \to (Q \to R)$ & $P \land Q \to R$ \\
0 & 0 & 0 & 1                 & 1                 \\
0 & 0 & 1 & 1                 & 1                 \\
0 & 1 & 0 & 1                 & 1                 \\
0 & 1 & 1 & 1                 & 1                 \\
1 & 0 & 0 & 1                 & 1                 \\
1 & 0 & 1 & 1                 & 1                 \\
1 & 1 & 0 & 0                 & 0                 \\
1 & 1 & 1 & 1                 & 1
\end{tabular}
\caption*{表 1.3 (5)}
\end{table} 

\paragraph{1.4}\label{1.4}
下列公式哪个是重言式,永假式和可满足的,并用代入规则(对重言式)或真值表来验证。\\

\quad (1) $P \to P$. \\

\quad (3) $(Q\to R)\to((P\lor Q)\to (P \lor R)).$  \\

\quad (5) $(P \to Q) \to (\lnot Q \to \lnot P).$ \\

\paragraph{解}

\subparagraph{(1)} 该公式是重言式。

\subparagraph{(3)} 该公式是重言式。真值表如表 1.4(3) 所示。

\begin{table}[!htb]
\centering
\begin{tabular}{ccccccccc}
$P$                                           & 0 & 0 & 0 & 0 & 1 & 1 & 1 & 1 \\
$Q$                                           & 0 & 0 & 1 & 1 & 0 & 0 & 1 & 1 \\
$R$                                           & 0 & 1 & 0 & 1 & 0 & 1 & 0 & 1 \\
$P \lor Q$                                  & 0 & 0 & 1 & 1 & 1 & 1 & 1 & 1 \\
$P \lor R$                                  & 0 & 1 & 0 & 1 & 1 & 1 & 1 & 1 \\
$(P \lor Q) \to (P \lor R)$                 & 1 & 1 & 0 & 1 & 1 & 1 & 1 & 1 \\
$Q \to R$                                   & 1 & 1 & 0 & 1 & 1 & 1 & 0 & 1 \\
$(Q \to R) \to ((P \lor Q) \to (P \lor R))$ & 1 & 1 & 1 & 1 & 1 & 1 & 1 & 1
\end{tabular}
\caption*{表 1.4 (3)}
\end{table}

\subparagraph{(5)} 该公式是重言式。真值表如表 1.4(5) 所示。

\begin{table}[!htb]
\centering
\begin{tabular}{ccccc}
$P$                                   & 0 & 0 & 1 & 1 \\
$Q$                                   & 0 & 1 & 0 & 1 \\
$P \to Q$                             & 1 & 1 & 0 & 1 \\
$\lnot Q \to \lnot P$                 & 1 & 1 & 0 & 1 \\
$(P \to Q) \to (\lnot Q \to \lnot P)$ & 1 & 1 & 1 & 1
\end{tabular}
\caption*{表 1.4 (5)}
\end{table}

\paragraph{1.5}\label{1.5}
形式化下列自然语句。\\

\quad (1) 他个子高而且很胖。 \\

\quad (3) 并非“他个子高或很胖”。\\

\quad (5) 他个子高或者他个子矮而很胖。\\

\quad (7) 如果水是清的,那么或者张三能见到池底或者他是个近视眼。\\

\paragraph{解} 
\subparagraph{(1)}$P\land Q. $ 其中$P: $他很高;$Q: $他很胖。
\subparagraph{(3)}$\lnot (P \lor Q).$ 定义同上。
\subparagraph{(5)}$P \lor (\lnot P \land Q).$ 定义同上。
\subparagraph{(7)}$P\to (Q \land R).$ 其中$P: $水是清的;$Q: $张三能看到池底;$R: $张三是近视眼。



\end{document} 