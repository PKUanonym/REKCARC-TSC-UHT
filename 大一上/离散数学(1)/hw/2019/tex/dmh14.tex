% UTF-8 encoding
% Compile with latex+dvipdfmx, pdflatex, xelatex or lualatex

\documentclass[UTF8]{ctexart}
\usepackage{graphicx}
\usepackage{amssymb}
\usepackage{amsmath}
\usepackage{subfigure}
\usepackage{geometry}
\usepackage{caption}

\newcommand{\true}{{\rm T}}
\newcommand{\false}{{\rm F}}
\newcommand{\snatural}{\mathbb{N}}
\newcommand{\sinteger}{\mathbb{Z}}
\newcommand{\srational}{\mathbb{Q}}
\newcommand{\sreal}{\mathbb{R}}
\newcommand{\card}{\rm{card}}
\newcommand{\modt}{\text{mod }}
\newcommand{\ran}{{\rm ran}}
\newcommand{\dom}{{\rm dom}}

\title{离散数学——第十五周作业}
\author{计83  刘轩奇  2018011025}
\date{2019.12.20}

\geometry{left=2.0cm, right=2.0cm, top=2.5cm, bottom=2.5cm}

\begin{document}

\maketitle

\paragraph{12.2} \label{12.2}
    用等势定义证明$[0,1] \approx [a,b], (a,b \in \sreal, a<b)$.

\paragraph{证}
    $f:[0,1] \rightarrow [a,b], f(x) = a+x(b-a)$是双射函数,从而$[0,1] \approx [a,b], (a,b \in \sreal, a<b)$

\paragraph{12.4} \label{12.4}
    写出$\snatural$的三个与$\snatural$等势的真子集。

\paragraph{答}
    正自然数集合$\snatural ^+ = \snatural - \{0\}$

    偶自然数集合$\snatural_{\text{even}} = \{x | ( \exists y) (y \in \snatural \land x = 2y) \}$

    奇自然数集合$\snatural_{\text{odd}} = \snatural^+ - \snatural_{\text{even}} = \{x | ( \exists y) (y \in \snatural \land x = 2y+1) \}$

\paragraph{12.7} \label{12.7}
    对于任意的基数$k,l$和无限基数$m$,如果$2 \le k \le m$且$2 \le l \le m$,证明

    (1) $k^m = 2^m$

    (2) $k^m = l^m$

\paragraph{证}
    (1) $$2^m \le k^m \le m^m \le 2^m$$
    $$\therefore k^m = 2^m$$

    (2) $k^m = 2^m$,同理$l^m = 2^m$,从而
    $$k^m = l^m$$

\paragraph{12.9} \label{12.9}
    证明平面上直角坐标系中的所有整数坐标点的集合使可数集。

\paragraph{证}
    设所有整数坐标点的集合为$A = \{(x,y) | x \in \sinteger \land y \in \sinteger \}$
    $$\card A = \card(\sinteger \times \sinteger) = \aleph_0 \cdot \aleph_0 = \aleph_0$$
    从而$A$是可数集。

\paragraph{12.10} \label{12.10}
    计算下列集合的基数。
    
    (1) $A = \{a,b,c\}$

    (2) $B = \{x|(\exists n)(n \in\snatural \land x = n^2)\}$

    (3) $D = \{x|(\exists n)(n \in\snatural \land x = n^5)\}$

    (4) $B \cap D$

    (5) $B \cup D$

    (6) $\snatural_\snatural$
    
    (7) $\sreal_\sreal$

\paragraph{解}
    (1) $\card A = 3$

    (2) $f: \snatural \to B, f(x) = x^2$是双射函数,则$\card B = \aleph_0$

    (3) $f: \snatural \to D, f(x) = x^5$是双射函数,则$\card D= \aleph_0$

    (4) 记$C = B \cap D$,显然$1 = 1^2 = 1^5$从而$1 \in C$;同时$1024 = 32^2 = 4^5$,则$1024 \in C$。另外易知$C$是自然数集$\snatural$的子集,则$\card C \le \aleph_0$。

    假设$C$是有限集,可令$a$是全序集$\langle \snatural, \le \rangle$上定义的$C$的最大元。则必有$a \ge 1024$。取$b = a^10 = (a^2)^5 = (a^5)^2 > a$,则$b \in C$,这与$a$是$C$的最大元相矛盾。从而$C$不是有限集,$\card C \ge \aleph_0$

    综上,$\card C = \aleph_0$

    (5) $\aleph_0 \le \card(B \cup D) \le  \aleph_0 + \aleph_0 = \aleph_0, \therefore \card(B \cup D) = \aleph_0$

    (6) $\card(\snatural_\snatural) = \aleph_0^{\aleph_0} = 2^{\aleph_0} = \aleph_1$

    (7) $\card(\sreal_\sreal) = \aleph_1 ^ {\aleph_1} = 2^{\aleph_1} = \aleph_2$

\end{document} 