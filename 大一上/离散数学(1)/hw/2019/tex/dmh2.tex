% UTF-8 encoding
% Compile with latex+dvipdfmx, pdflatex, xelatex or lualatex

\documentclass[hyperref, UTF8]{ctexart}
\usepackage{graphicx}
\usepackage{amssymb}
\usepackage{amsmath}
\usepackage{subfigure}
\usepackage{geometry}
\usepackage{caption}

\newcommand{\true}{{\rm T}}
\newcommand{\false}{{\rm F}}

\title{离散数学——第三周作业}
\author{计83  刘轩奇  2018011025}
\date{2019.09.27}

\geometry{left=2.0cm, right=2.0cm, top=2.5cm, bottom=2.5cm}
\linespread{0.7}
\begin{document}

\maketitle

\paragraph{1.6}\label{1.6}
将下列公式写成波兰式和逆波兰式。\\

(1) $ P \rightarrow Q \lor R \lor S $ \\

(2) $ P \land \lnot R \leftrightarrow P \lor Q $ \\

(3) $ \lnot \lnot P \lor (W \land R) \lor \lnot Q $

\paragraph{解}

(1) 波兰式 $ \rightarrow P \lor \lor QRS $ ,逆波兰式 $ PQR \lor S \lor \rightarrow $。\\

(2) 波兰式 $ \leftrightarrow \land P \lnot R \lor PQ $ ,逆波兰式 $ PR \lnot \land PQ \lor \leftrightarrow $。\\

(3) 波兰式 $ \lor \lor \lnot \lnot P \land WR \lnot Q $ ,逆波兰式 $ P \lnot \lnot WR \land \lor Q \lnot \lor $。 \\

\paragraph{2.1}\label{2.1}
证明下列等值公式。\\

(1) $ P \rightarrow (Q \land R)=(P \rightarrow Q) \land (P \rightarrow R) $ \\

(3) $ ((P \rightarrow \lnot Q) \rightarrow (Q \rightarrow \lnot P)) \land R=R $ \\

(5) $ P \rightarrow (Q \rightarrow R)=(P \land Q) \rightarrow R $

\paragraph{证}

(1)
\begin{align*}
P \rightarrow (Q \land R) &= \lnot P \lor (Q \land R) \\
&=( \lnot P \lor Q) \land ( \lnot P \lor R) \\ &= (P \rightarrow Q) \land (P \rightarrow R)
\end{align*}

(3)
\begin{align*}
((P \rightarrow \lnot Q) \rightarrow (Q \rightarrow \lnot P)) \land R &=(( \lnot P \lor \lnot Q) \rightarrow ( \lnot Q \lor \lnot P)) \land R \\ &=R
\end{align*}

(5)
\begin{align*}
P \rightarrow (Q \rightarrow R)&=P \rightarrow ( \lnot Q \lor R) \\ &= \lnot P \lor ( \lnot Q \lor R) \\ &=( \lnot P \lor \lnot Q) \lor R \\ &= \lnot (P \land Q) \lor R \\ &=(P \land Q) \rightarrow R
\end{align*}

\paragraph{2.2}\label{2.2}
由下列真值表,分别从$\true$和$\false$来列写出$A, B$和$C$的表达式,并分别以符号$m_i$和$M_i$表示。

\begin{table}[!htb]
\centering
\begin{tabular}{ccccc}
\hline
$P$ & $Q$ & $A$ & $B$ & $C$ \\
\hline
$\false$ & $\false$ & $\true$ & $\true$ & $\true$ \\
$\false$ & $\true$ & $\true$ & $\false$ & $\false$ \\
$\true$ & $\false$ & $\true$ & $\false$ & $\false$ \\
$\true$ & $\true$ & $\false$ & $\true$ & $\false$ \\
\hline
\end{tabular}
\end{table}

\paragraph{解}

\begin{align*}
A
  & = ( \lnot P \land \lnot Q) \lor ( \lnot P \land Q) \lor (P \land \lnot Q) \\
  & = m_0 \lor m_1 \lor m_2 \\
  & = ( \lnot Q \lor \lnot P) \\
  & = M_0 \\
B
  & = ( \lnot P \land \lnot Q) \lor (P \land Q) \\
  & = m_0 \lor m_3 \\
  & = (P \lor \lnot Q) \land ( \lnot P \lor Q) \\
  & = M_1 \land M_2 \\
C
  & = ( \lnot P \land \lnot Q) \\
  & = m_0 \\
  & = (P \lor \lnot Q) \land ( \lnot P \lor Q) \land ( \lnot P \lor \lnot Q) \\
  & =M_0 \land M_1 \land M_2
\end{align*}

\paragraph{2.3}\label{2.3}
用$\uparrow$和$\downarrow$分别表示出$ \lnot , \land , \lor , \rightarrow $和$\leftrightarrow $。

\paragraph{解}

(1) 用$\uparrow$
\begin{align*}
\lnot P &= \lnot (P \land P) \\&= P \uparrow P \\
P \land Q &= \lnot ( \lnot (P \land Q)) \\&= \lnot (P \uparrow Q) \\&= (P \uparrow Q) \uparrow (P \uparrow Q) \\
P \lor Q &= \lnot ( \lnot P \land \lnot Q) \\&= \lnot P \uparrow \lnot Q \\&= (P \uparrow P) \uparrow (Q \uparrow Q) \\
P \rightarrow Q &= \lnot P \lor Q \\&= ( \lnot P \uparrow \lnot P) \uparrow (Q \uparrow Q) \\&= P \uparrow (Q \uparrow Q) \\
P \leftrightarrow Q &= (P \land Q) \lor ( \lnot P \land \lnot Q) \\&= \lnot \lnot ((P \land Q) \lor ( \lnot P \land \lnot Q)) \\&= \lnot ( \lnot (P \land Q) \lor \lnot ( \lnot P \land \lnot Q)) \\&= \lnot (P \uparrow Q) \land ( \lnot P \uparrow Q) \\&= (P \uparrow Q) \uparrow ( \lnot P \uparrow \lnot Q) \\&= (P \uparrow Q) \uparrow ((P \uparrow P) \uparrow (Q \uparrow Q))
\end{align*}
(2) 用$\downarrow$
\begin{align*}
\lnot P &= \lnot (P \lor P) \\&= P \downarrow P \\
P \land Q &= \lnot ( \lnot P \lor \lnot Q) \\&= \lnot P \downarrow \lnot Q \\&= (P \downarrow P) \downarrow (Q \downarrow Q) \\
P \lor Q &= \lnot \lnot (P \lor Q) \\&= \lnot (P \downarrow Q) \\&= (P \downarrow Q) \downarrow (P \downarrow Q) \\
P \rightarrow Q &= \lnot P \lor Q \\&= ( \lnot P \downarrow Q) \downarrow ( \lnot P \downarrow Q) \\&= =((P \downarrow P) \downarrow Q) \downarrow ((P \downarrow P) \downarrow Q) \\
P \leftrightarrow Q &= (P \land Q) \lor ( \lnot P \land \lnot Q) \\&= ( \lnot ( \lnot P \lor \lnot Q) \lor ( \lnot (P \lor Q)) \\&= ( \lnot P \downarrow \lnot Q) \lor (P \downarrow Q) \\&= \lnot \lnot (( \lnot P \downarrow Q) \lor (P \downarrow Q)) \\&= \lnot (( \lnot P \downarrow \lnot Q) \downarrow (P \downarrow Q)) \\&= \lnot (((P \downarrow P) \downarrow (Q \downarrow Q)) \downarrow (P \downarrow Q)) \\&= (((P \downarrow P) \downarrow (Q \downarrow Q)) \downarrow (P \downarrow Q)) \downarrow (((P \downarrow P) \downarrow (Q \downarrow Q)) \downarrow (P \downarrow Q))
\end{align*}

\paragraph{2.4}\label{2.4}
证明 \\

(1) $A\rightarrow B$与$B^*\rightarrow A^*$同永真、同可满足。\\

(2) $A\leftrightarrow B$与$A^*\leftrightarrow B^*$同永真、同可满足。\\

\paragraph{证}

(1)
$$
A\rightarrow B=\lnot B \rightarrow \lnot A = B^{*-} \rightarrow A^{*-}
$$

而$B^{*-} \rightarrow A^{*-}$与$B^* \rightarrow A^*$同永真、同可满足。\\

$\therefore$$A\rightarrow B$与$B^* \rightarrow A^*$同永真、同可满足。\\

(2)
\begin{align*}
A\leftrightarrow B &= \lnot \lnot (A \leftrightarrow B) \\&= \lnot (A \leftrightarrow \lnot B) \\&= ( \lnot A) \leftrightarrow ( \lnot B) \\ &= (A^{*-})\leftrightarrow (B^{*-})
\end{align*}

而$(A^{*-})\leftrightarrow (B^{*-})$与$A^* \leftrightarrow B^*$同永真、同可满足。\\

$\therefore$$A\leftrightarrow B$与$A^* \leftrightarrow B^*$同永真、同可满足。

\end{document} 